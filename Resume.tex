%%%%%%%%%%%%%%%%%%%%%%%%%%%%%%%%%%%%%%%%%%
%% Simple LaTeX Resume by Biswajit Paria %
%%%%%%%%%%%%%%%%%%%%%%%%%%%%%%%%%%%%%%%%%%

%%%%%%%%%%%%%%%%%%%%%%%%%%%%% Preamble %%%%%%%%%%%%%%%%%%%%%%%%%%%%%%%%%%%%%%%%%

\documentclass[10pt]{article}

\usepackage{graphicx}
\usepackage{tabto}
\usepackage{amsmath}
\usepackage{amssymb}
\usepackage{algorithm, algorithmicx}
\usepackage{algpseudocode}
\usepackage{xcolor}
\usepackage{subcaption}
\usepackage{booktabs}
\usepackage{url}
\usepackage{parskip}
\usepackage[a4paper, total={7in, 10in}, top=1in]{geometry}
\usepackage{hyperref}

\pagenumbering{gobble}

\setlength{\parindent}{0pt}
\setlength{\itemsep}{0pt}
  \setlength{\parskip}{0pt}
  \setlength{\parsep}{0pt}
\renewcommand{\baselinestretch}{1}

\newcommand{\heading}[1]{
 {\large \textbf{#1}}
  \vspace{0.4em}
  \hrule
  \vspace{0.4em}
}

\newcommand{\EntryGap}{\vspace{0.4  cm}}
\newcommand{\SmallEntryGap}{\vspace{0.2cm}}
\newcommand{\mdot}{$\ \ \cdot\ \ $}

\newcommand{\indentedpar}[1]{
  \hangindent=1cm \hangafter=0 #1
}

\newenvironment{compactitemize}
{ \begin{itemize}
  \setlength{\itemsep}{0pt}
  \setlength{\parskip}{0pt}
  \setlength{\parsep}{0pt}}
{ \end{itemize}}


%%%%%%%%%%%%%%%%%%%%%%%%%%%%% Document %%%%%%%%%%%%%%%%%%%%%%%%%%%%%%%%%%%%%%%%%

\begin{document}



%%%%%%%%%%%%%%%%%%%%%%%%%%%%% Title %%%%%%%%%%%%%%%%%%%%%%%%%%%%%%%%%%%%%%%%%%%%

{\Large \textbf{Subham Sekhar Sahoo}} \hfill subbham@iitkgp.ac.in\\
3rd Year Undergraduate Student\hfill subbham@gmail.com\\
Electrical Engineering Department \hfill \url{https://shakeh3r.github.io}\\
Indian Institute of Technology, Kharagpur \hfill  +91-9002966333  
\EntryGap


%%%%%%%%%%%%%%%%%%%%%%%%%%%%% EDUCATION  %%%%%%%%%%%%%%%%%%%%%%%%%%%%%%%%%%%%%%%

\heading{Education}

\textbf{Indian Institute of Technology Kharagpur, India} \hfill Jul 2015 - present\\
Electrical Engineering major and a Computer Science minor\\
GPA \textbf{8.3/10.0}

\SmallEntryGap
\textbf{DAV Public School, CDA} \hfill Apr 2012 - Apr 2014\\
All India Senior School Certificate Examination (AISSCE), CBSE board\\
Percentage score \textbf{91.2\%}

\SmallEntryGap
\textbf{DAV Public School, CDA} \hfill Apr 2007 - Apr 2012\\
All India Secondary School Examination (AISSE), CBSE board\\
GPA \textbf{10.0/10.0}

\EntryGap
\heading{Interests}

Interested in Machine Learning, Reinforcement Learning and related fields. Enjoys working on own initiative or in a team and has a practical approach to problem solving with a drive to see things through to completion.

%%%%%%%%%%%%%%%%%%%%%%%%%%%%%%%% PAPERS
\EntryGap

\heading{Papers}
  
  \textbf{Subham Sekhar Sahoo} and Georg Martius.  \emph{"Learning Equation With A Deep Network"}\\
  Association for the Advancement of Artificial Intelligence \emph{(AAAI-18)}
  (Submitted)
%%%%%%%%%%%%%%%%%%%%%%%%%%%%%%%%%%% INTERNSHIP
\EntryGap

\heading{Internships and Projects}

\textbf{Max Planck Institute for Intelligent Systems, Tuebingen, Germany} \hfill May 2017 - July 2017\\
\textbf{\emph{Equation Learner}}\\
Advisor: Dr. Georg Martius (Group Leader- Autonomous Learning Group)

\SmallEntryGap

\indentedpar{
  $\bullet$ The model (\emph{EQL - Equation Learner}), an end-to-end differentiable feed-forward network, is a multi-layered feed-forward network with computational units specifically designed for the \emph{extrapolation regression} tasks for systems that can be described by real-valued analytic expression, e.g. mechanical systems such as a pendulum or a robotic arm.\\
  $\bullet$ Due to sparsity regularization concise interpretable expressions are obtained and the source equation of the data (with noise) is often learnt. \\
  $\bullet$ Deisigned the most significant feature of this network- making it learn the source equations with \emph{division} operation. For this a special kind of cost function was formulated which can be used to \emph{learn equations with poles} in general.
}

\SmallEntryGap

\textbf{Max Planck Institute for Intelligent Systems, Tuebingen, Germany} \hfill July 2017 - present\\
\textbf{\emph{Deep Reinforcement Learning For Locomotion Of A Musco-skeletal Robot}}\\
Advisor: Dr. Georg Martius (Group Leader- Autonomous Learning Group)

\SmallEntryGap

\indentedpar{
  $\bullet$ Aim - to develop a controller to enable a physiologically-based human model with 41 dimensional state space and 18 dimensional action space, to navigate a complex obstacle course as quickly as possible.\\
  $\bullet$ Potential obstacles include external obstacles like steps, or a slippery floor, along with internal obstacles like muscle weakness or motor noise. The state space is a 41 dimensional vector and the action space is 18 dimensional. \\
}

\SmallEntryGap

\textbf{AgNEXT} \hfill October 2016 - December 2016\\
\textbf{\emph{Crop Disease Detection Using Deep Learning}}\\
Advisors: Dr. Sangram Ganguly (Research Scientist, NASA Ames Research Center, California, USA) \& \\
\hspace*{15mm} Dr. Mrigank Sharad (Asst. Professor, IIT Kharagpur)

\SmallEntryGap

\indentedpar{
Part of the Artificial Intelligence team that created an app meant for people involved in agriculture that :-\\
$\bullet$ Detects crop disease (if any) based on its leaf's image. A Convolutional Neural Network was trained on a leaf-image dataset with pre-processng. The prediction accuracy was in the range \textbf{97-98\%}.\\
$\bullet$ Alerts the user if the time is right to use pesticides by capturing the image of a sticky board and then analysing  the number of insects, crop type and weather conditions.
}

\SmallEntryGap

\textbf{Autonomous Ground Vehicle Research Group, IIT Kharagpur} \hfill October 2016 - December 2016\\
\textbf{\emph{Autonomous Car}}\\
Adivisor: Dr. Debasish Chakraborty (Professor, IIT Kharagpur)

\SmallEntryGap

\indentedpar{
$\bullet$ Designed a CNN based steering angle predictor. The dataset was made available by \href{https://github.com/udacity/self-driving-car}{Udacity}. Made a training-validation split of 85:15 and preprocessed the imbalanced data with excessive 0 steering angle. The best set of weights was selected with the best validation score.\\
$\bullet$ Designed a generative model to simulate driving, as proposed in \href{https://arxiv.org/pdf/1608.01230.pdf}{Learning a Driving Simulator} by Comma.ai. Variational auto-encoder was used to embed road frames. A transition model (action conditioned RNN) was then learnt in the embedded space to produce realisitic looking video for several frames.\\
$\bullet$ Designed a DQN based controller with experience replay to navigate a 4 wheeled bot with lidar in an environment with dynamic obstacles using Deep-Reinforcement Learning.
}

\SmallEntryGap


\textbf{Autonomous Ground Vehicle Research Group, IIT Kharagpur} \hfill May 2016 - October 2016\\
\textbf{\emph{Eklavya 4.0}}\\
Advisor: Dr. Debasish Chakraborty (Professor, IIT Kharagpur)

\SmallEntryGap

\indentedpar{
$\bullet$ Implemented \emph{SBPL Planner} for Lane Navigation. The motion primitives were generated keeping in mind the physical constraints of the bot.\\
$\bullet$ Implemented \emph{Waypoint Navigation} module and basic motion planning algorithms on Robot Operating System(ROS).\\
$\bullet$ Prepared a Gazebo simulation for this bot.
}

\SmallEntryGap


\textbf{Indian Institute of Technology Kharagpur, India} \hfill January 2017-present\\
\textbf{\emph{Novel Architectures Of GANs For Semi-supervised \& Un-supervised Learning}}\\
Advisor: Dr. Pabitra Kumar Biswas (H.O.D, Electronics and Electrical engineering department, IIT Kharagpur)

\SmallEntryGap

\indentedpar{
Improving existing \emph{Generative Adverserial Network} architectures for achieving better classification accuracy in case of semi-supervised and un-supervised learning is the objective of this project.
}

%%%%%%%%%%%%%%%%%%%%%%%%%%%%%%%% ACADEMIC HONORS & AWARDS %%%%%%%%%%%%%%%%%%%%%%


\EntryGap
\heading{Academic Honors and Awards}

 \textbf{All India Joint Entrance Exam} \hfill 2015\\
  Secured a rank in \textbf{\emph{top 0.5 percentile}} out of 1.5 million students who took the JEE exam.\\
  
 \textbf{Kishore Vaigyanik Protsahan Yojana (KVPY) Scholar} \hfill 2014\\
  By Dept. of Science and Technology, Govt. of India for exceptional aptitude in basic sciences.\\
  Secured a rank in \textbf{\emph{top 0.3 percentile}} out of 0.1 million students who took the exam.\\
  
 \textbf{Scholar's Badge}\hfill 2012\\
     Awarded with a scholar's badge in high school for securing a \textbf{\emph{GPA 10.0}} for 3 running years.


%%%%%%%%%%%%%%%%%%%%%%%%%%%%%%%%%%Certifications%%%%%%%%%%%%%%%%%%%%%%%%%%%%%%%

\EntryGap
\heading{Summer Schools}
\textbf{Machine Learning Summer School (MLSS'17)} \hfill 19\textsuperscript{th} June 2017 - 30\textsuperscript{th} June 2017\\
organised at - Max Planck Institute, Tuebingen, Germany.\\
Attended \href{http://mlss.tuebingen.mpg.de/}{MLSS-2017} held at Max Planck Institute for Intelligent Systems,
Tuebingen. The summer school went on for two weeks and it hoisted a number of speakers
excelling in the field of Optimisation, Learning Theory, Bayesian Inference, Causality, sub modularity, Reinforcement
Learning and Robotics.  


%%%%%%%%%%%%%%%%%%%%%%%%%%%%%%%%%%%%%% SKILLS %%%%%%%%%%%%%%%%%%%%%%%%%%%%%%%%%%

\EntryGap
\heading{Technical Skills}
\SmallEntryGap
\textbf{Proficient:} C, C++, ROS(Robot Operating System), Gazebo, Python, Matlab, Tensorflow, Theano, Numpy\\
\textbf{Familiar:} Bash, HTML, Javascript, Scikit-learn, Keras, Lasagne, Nltk



%%%%%%%%%%%%%%%%%%%%%%%%%% RELEVANT COURSES   %%%%%%%%%%%%%%%%%%%%%%%%%%%%%%%%%%


\EntryGap
\heading{Relevant Coursework}

\SmallEntryGap
\begin{tabular}{lll}
 $\bullet$ Algorithms\\
 $\bullet$ Machine Learning \\
 $\bullet$ Deep Learning\\
 $\bullet$ Reinforcement Learning\\
 $\bullet$ Data Analytics*\\
\end{tabular}

\SmallEntryGap
\SmallEntryGap
\SmallEntryGap
\emph{* Courses taken in the current semester}
% \textbf{Others}\\
% Algorithms-I \& II \mdot Discrete Mathematics \mdot Parallel and Distributed Algorithms 
% \mdot Selected Topics in Algorithms  \mdot Computational Statistics
% \mdot Advanced Graph Theory \mdot Database Management Systems \mdot Operating Systems
% \mdot Computer Networks \mdot Computer Organization and Architecture \mdot Theory of Computation
% \mdot Operations Research \mdot High Performance Computer Architecture \mdot Distributed Systems
% \mdot Cryptography

\end{document}
